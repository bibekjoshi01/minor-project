\subsection{System Architecture and Block Diagram}

The overall system architecture is designed as a tightly coupled hardware-software control system implemented on a single embedded platform. The architecture follows a modular structure in which sensing, decision making, and actuation are clearly separated, allowing baseline control strategies and reinforcement learning-based alignment to share the same hardware and execution pipeline.

\begin{figure}[h!]
    \centering
    \includegraphics[width=1\textwidth]{images/system_block_diagram.png}
    \caption{High-Level System Block Diagram}
    \label{fig:system_block}
\end{figure}

\subsubsection{High-Level Hardware Architecture}
The hardware architecture of the proposed system is illustrated in Figure~\ref{fig:system_block}. The ESP32 microcontroller forms the central processing unit and coordinates all sensing, control, and communication tasks.

The major hardware components are:
\begin{itemize}
    \item \textbf{ESP32 Microcontroller}: Acts as the central controller, responsible for RSSI acquisition, control logic execution, and motor command generation
    \item \textbf{Directional Antenna}: Mounted on a mechanical platform and connected to the ESP32 radio subsystem. It receives wireless packets from the transmitter, from which RSSI values are extracted
    \item \textbf{Stepper Motor}: Provides controlled rotational motion of the directional antenna in the azimuth plan and tilt plane
    \item \textbf{Motor Driver (A4988)}: Interfaces between the ESP32 and the stepper motor, translating low-power digital control signals into the high-current waveforms required to drive the motor coils
    \item \textbf{Power Supply Unit}: Provides regulated power to the ESP32 and a separate higher-current supply rail for the stepper motor to prevent electrical noise from affecting RF measurements
\end{itemize}

The ESP32 communicates with the motor driver using digital GPIO signals (STEP, DIRECTION, and ENABLE). The motor driver operates as a unidirectional actuator interface; no feedback or sensing signals are returned from the motor driver to the ESP32. All motion-related state is maintained internally by the controller.

\subsubsection{High-Level Software Architecture}
The software architecture mirrors the hardware modularity and is structured into four primary functional blocks executed on the ESP32:
\begin{itemize}
    \item \textbf{RSSI Acquisition Module}: Handles wireless packet reception and extracts RSSI values from the ESP32 radio subsystem. A lightweight filtering stage is applied to reduce short-term fluctuations
    
    \item \textbf{State Encoding Module}: Converts raw RSSI measurements and antenna orientation into a compact, discrete state representation suitable for control and reinforcement learning
    
    \item \textbf{Decision Module}: Implements the selected alignment strategy. This may correspond to a baseline method (scanning, hill-climbing) or a reinforcement learning policy based on Q-table lookup
    
    \item \textbf{Antenna Control Module}: Translates the selected action into motor commands by generating appropriate step and direction signals for the motor driver
\end{itemize}

These modules are executed sequentially within a timed control loop, ensuring deterministic behavior and repeatable experimental conditions.


\subsubsection{Data Flow and Control Loop}

Figure~\ref{fig:system_block} represents the primary data and control flow of the system. The flow proceeds as follows:

\begin{itemize}
    \item The ESP32 receives wireless packets via the directional antenna and extracts the corresponding RSSI values
    \item Filtered RSSI measurements are passed to the state encoding block along with the current antenna orientation
    \item The decision module evaluates the current state and selects an action according to the active alignment strategy
    \item The antenna control module issues step and direction commands to the motor driver
    \item The stepper motor rotates the antenna, and the system waits for mechanical stabilization
    \item The updated antenna orientation affects subsequent RSSI measurements, closing the control loop
\end{itemize}

This closed-loop operation continues until convergence criteria are met or the system is manually terminated.
