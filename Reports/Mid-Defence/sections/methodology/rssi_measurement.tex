\subsection{RSSI Measurement and Signal Conditioning}

The performance of the autonomous antenna alignment system relies heavily on accurate measurement of the received signal strength. Received Signal Strength Indicator (RSSI) is a hardware-reported metric representing the power of incoming radio packets, typically expressed in dBm. Higher RSSI values correspond to stronger received signals, providing a direct indication of antenna alignment quality. 

Due to inherent noise in wireless channels caused by multipath propagation, fading, and environmental interference, raw RSSI values must be conditioned before being used for control or reinforcement learning.

\subsubsection{RSSI Sampling}

RSSI is obtained from the ESP32's integrated WiFi radio, corresponding to the power of incoming packets received by the directional antenna. Each measurement is associated with a specific antenna orientation in both the azimuth and tilt planes. To capture sufficient temporal dynamics while avoiding MCU overload, a sampling rate of 5--20 Hz is employed. 

\subsubsection{Noise Mitigation and Filtering}

RSSI measurements are inherently volatile and can fluctuate significantly even when the antenna orientation remains fixed. To ensure robust control and learning, the following techniques are applied:
\begin{itemize}
    \item \textbf{Short-term averaging:} Multiple consecutive RSSI readings at the same orientation are averaged to reduce random noise.
    \item \textbf{Moving average filter:} A sliding window of recent RSSI samples is maintained to smooth out transient dips or spikes.
\end{itemize}

These techniques ensure that the control algorithm and RL agent react to meaningful signal trends rather than instantaneous fluctuations.

\subsubsection{State Feature Construction}

The RL agent requires a compact representation of the environment in the form of a state vector. The state vector is constructed using:
\begin{itemize}
    \item \textbf{Antenna orientation:} Both azimuth and tilt angles are discretized into fixed steps to limit the size of the state space.
    \item \textbf{RSSI trend:} The rate of change of filtered RSSI values is classified into three discrete states:
        \begin{itemize}
            \item \textbf{Increasing:} $\Delta RSSI > +1$ dB
            \item \textbf{Stable:} $-1 \leq \Delta RSSI \leq +1$ dB
            \item \textbf{Decreasing:} $\Delta RSSI < -1$ dB
        \end{itemize}
\end{itemize}

The resulting state vector is expressed as:
\[
\text{state} = (\text{azimuth\_index}, \text{tilt\_index}, \text{RSSI\_trend})
\]
This representation captures both the current antenna orientation and short-term signal behavior, enabling the RL agent to make informed alignment decisions without requiring complex channel estimation.

\subsubsection{Data Logging}

For debugging, performance evaluation, or offline analysis, RSSI measurements along with corresponding timestamps and antenna orientations can be logged temporarily on the ESP32 or transmitted to a host system. This facilitates visualization of signal trends and aids in parameter tuning for the RL agent.