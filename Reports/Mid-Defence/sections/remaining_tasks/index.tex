\newpage
\section{Remaining Works}

While the reinforcement learning--based antenna alignment strategy has been
successfully developed and validated in simulation, several implementation
tasks remain before final system deployment. These tasks primarily involve
hardware integration, embedded firmware deployment, and system-level
validation.

\subsection{Hardware Integration and Mechanical Refinement}

The next phase of work will focus on finalizing the mechanical and electrical
interfaces of the antenna positioning system. Planned improvements include
minor refinements to the stator and slip-ring assembly to improve electrical
contact reliability under rotation, as well as adjustments to mechanical
clearances to increase tolerance to shaft misalignment and wear. These changes
are intended to ensure stable long-term operation during continuous antenna
motion.

\subsection{Embedded Firmware Integration}

A key remaining task is the integration of the trained reinforcement learning
policy into the MCU firmware. This involves transferring the learned Q-table
from the simulation environment to the embedded platform and implementing a
deterministic inference loop for real-time action selection.

Additional firmware-level enhancements will include:
\begin{itemize}
    \item Coordinated control of azimuth and tilt actuators
    \item Non-blocking motor command execution and timing management
    \item RSSI sampling synchronized with mechanical stabilization
\end{itemize}

These changes are required to ensure that control decisions are based on valid
and repeatable signal measurements.

\subsection{System Testing and Validation}

Following hardware and firmware integration, the complete system will undergo
experimental validation. This phase will evaluate alignment accuracy,
convergence time, and stability under real wireless channel conditions. The
results will be compared against baseline scanning strategies previously
evaluated in simulation to assess the effectiveness of the reinforcement
learning approach in practice.

This final validation stage will confirm the feasibility of the proposed method
for real-world directional antenna alignment.
