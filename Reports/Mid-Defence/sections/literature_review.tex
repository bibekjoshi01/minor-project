\newpage
\section{LITERATURE REVIEW}

Wireless communication has evolved significantly since the pioneering experiments of Marconi in the early 1900s, becoming a fundamental component of modern communication systems~\cite{ref3}. 
Advancements in wireless networks have been driven by the increasing demand for mobility, higher data rates, and efficient spectrum utilization~\cite{ref4}. 
Antennas remain one of the most critical components influencing communication range, throughput, and interference characteristics~\cite{ref5}. 
Modern wireless systems employ a wide variety of technologies including WiFi (IEEE 802.11), Bluetooth, NFC, LTE, 5G, and LoRa, each optimized for different ranges, data rates, and energy requirements. 
Despite these advances, reliable long-distance communication continues to face challenges due to signal attenuation, interference, and environmental dynamics.

Antennas are broadly classified as omnidirectional and directional. 
Omnidirectional antennas provide wide coverage but suffer from lower gain and higher susceptibility to interference. 
In contrast, directional antennas concentrate energy in specific directions, offering higher gain, extended range, reduced interference, and improved energy efficiency~\cite{ref6}. 
Directional antennas are therefore widely adopted in long-range networks such as Wireless Sensor Networks (WSNs), UAV communication systems, IoT deployments, and mobile ad-hoc networks. 
However, their performance depends critically on accurate alignment between transmitter and receiver. 
For directional antennas to function effectively, precise orientation is essential. 
Traditional antenna alignment relies on GPS-based coordinate tracking methods or signal tracking control methods (TCM)~\cite{ref7,ref8}. 
Other control-theoretic methods include imitation human intelligence control approaches, which attempt to mimic expert decision-making for antenna tracking~\cite{ref9}. 
Servo tracking systems have also been designed to enable precise mechanical alignment of vehicle-mounted antennas, demonstrating early practical implementations of automated antenna control~\cite{ref10}. 
These methods can improve performance over simple feedback-based systems but often require complex modelling. 
They suffer from limitations including GPS instability, high hardware cost, and reduced reliability in dynamic environments. 
In the absence of reliable position information, alignment must be performed using signal feedback, particularly the Received Signal Strength Indicator (RSSI). 
RSSI-based alignment allows the antenna to adjust orientation to maximize signal strength without explicit knowledge of node positions~\cite{ref11,ref12}.

Early RSSI-based approaches employed exhaustive scanning of antenna orientations to locate the direction that maximizes RSSI. 
While simple, such methods are inefficient and unsuitable for real-time systems due to high latency and energy consumption~\cite{ref11,ref12}. 
More advanced methods incorporate estimation and control techniques including Kalman filters, unscented Kalman filters, and fuzzy logic controllers, often fusing RSSI with partial GPS data~\cite{ref13}. 
Autonomous antenna systems capable of self-orientation have also been explored to enhance wireless range and signal quality without human intervention~\cite{ref14}. 
These approaches improve tracking accuracy but require additional sensors and accurate system models, making them impractical for low-cost embedded platforms. 
A notable control-theoretic approach was introduced by Koru et al.~\cite{ref11}, who developed a nonlinear distributed RSSI feedback control law for GPS-free alignment of UAV-mounted directional antennas. 
Their method achieves near-global convergence under specific assumptions but depends on accurate estimation of RSSI partial derivatives, which is difficult on low-cost hardware.

Recent research has explored the application of machine learning (ML) to antenna alignment problems where analytical models are complex or unavailable. 
Supervised learning approaches using MLP, CNN, and LSTM architectures have been shown to achieve high accuracy in predicting antenna alignment from S-parameter measurements~\cite{ref15}. 
Also, studies have surveyed AI-based beamforming and management techniques for 5G and 6G systems, highlighting the potential of learning-based approaches for optimizing antenna orientation and beam alignment~\cite{ref16}. 
However, these methods require large labelled datasets and controlled environments.
Reinforcement Learning (RL) offers a model-free alternative, particularly suited for control problems involving unknown environments. 
RL-based approaches have been applied to antenna orientation, channel selection, beam selection, and power control~\cite{ref17}. 
Learning-based beam alignment methods for mmWave and 5G systems using DQN have demonstrated improved performance in highly dynamic environments~\cite{ref17}.

Multi-agent RL frameworks have also been investigated for antenna tilt optimization in cellular networks, enabling cooperative learning among neighbouring cells~\cite{ref18}. 
However, most existing RL-based methods target high-end systems with antenna arrays, beamforming, continuous state spaces, and substantial computational resources, making them unsuitable for low-cost embedded platforms. 

While significant progress has been made in antenna alignment using RSSI-based control, filtering techniques, and learning-based methods, current solutions exhibit at least one of the following limitations:

\begin{itemize}
    \item Dependence on GPS or additional sensors
    \item Requirement of accurate system models and derivative estimation
    \item High computational complexity unsuitable for low-cost embedded devices
    \item Focus on antenna arrays and beamforming rather than mechanically steerable single antennas
\end{itemize}

This creates a clear research gap for a low-cost, RSSI-driven, reinforcement learning-based antenna alignment system capable of real-time operation on resource-constrained hardware.

\setlength{\LTpre}{1cm}   % no space before table
\setlength{\LTpost}{1cm}  % no space after table
\renewcommand{\arraystretch}{1.4}
\setlength{\tabcolsep}{8pt}

\begin{longtable}{|
    >{\raggedright\arraybackslash}p{4.4cm} |
    >{\raggedright\arraybackslash}p{4.4cm} |
    >{\raggedright\arraybackslash}p{4.4cm} |}

\caption{Summary of Literature Review}
\label{tab:literature-review}
\vspace{0.2cm} 
\hline
\multicolumn{1}{|c|}{\parbox[c][2em][c]{4cm}{\centering \textbf{Title}}} &
\multicolumn{1}{c|}{\parbox[c][2em][c]{4.5cm}{\centering \textbf{Methodology}}} &
\multicolumn{1}{c|}{\parbox[c][2em][c]{5.5cm}{\centering \textbf{Key Inferences}}} \\
\hline

\textit{RSSI-Based Distributed Control to Align Directional Antenna Pairs for UAV Communication} (2024) \newline
Y. Wan et al.
&
Control-theoretic approach using nonlinear static state feedback. RSSI partial derivatives estimated via antenna motion.
&
Demonstrates GPS-free RSSI-based antenna alignment with provable convergence. Strong theory, but requires accurate derivative estimation, difficult on low-cost embedded hardware.
\\ \hline

\textit{Received Signal Strength Indicator-Based Decentralised Control for Robust Long-Range Aerial Networking Using Directional Antennas} (2017) \newline
Yan et al.
&
Fusion of RSSI and GPS using Unscented Kalman Filter and fuzzy logic.
&
Improves robustness but depends on additional sensors and antenna models. Not suitable for minimal hardware systems.
\\ \hline

\textit{Self-Orientation of Directional Antennas Assisted by Mobile Robots} (2012) \newline
Min et al.
&
Pattern-based RSSI scanning using pan-tilt antenna mounts.
&
Validates RSSI-only alignment, but exhaustive scanning is slow and unsuitable for real-time systems.
\\ \hline

\textit{SmartAntenna: Enhancing Wireless Range with Autonomous Orientation} (2024) \newline
Swann et al.
&
Servo-mounted antenna using RSSI feedback and heuristic search.
&
Demonstrates low-cost feasibility, but performance degrades in noisy environments and lacks adaptive learning.
\\ \hline

\textit{Comparative Analysis of Machine Learning Algorithms for Antenna Alignments} (2024) \newline
Shakya et al.
&
Supervised ML (MLP, CNN, LSTM) trained on S-parameters.
&
High accuracy in controlled environments, but requires large datasets and offline training.
\\ \hline

\textit{Learning-Based Beam Alignment for Uplink mmWave UAVs} (2023) \newline
Susarla et al.
&
Deep Q-Network for beam alignment using antenna arrays.
&
Effective in mmWave systems, but unsuitable for single-antenna, low-cost embedded platforms.
\\ \hline

\end{longtable}
