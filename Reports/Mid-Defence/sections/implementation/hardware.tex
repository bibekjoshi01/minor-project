\subsection{Hardware Design}

The hardware architecture was designed to support continuous azimuth rotation,
independent elevation control, and stable RSSI sampling under motor actuation.
Emphasis was placed on mechanical rigidity, electrical noise isolation, and
modular separation between power, logic, and actuation subsystems.

\subsubsection{Mechanical Configuration and Structural Design}

The physical architecture is organized into a vertical stack consisting of a Power Deck, a Logic Deck, and a rotating Top Floor.

\textbf{Stationary Base (Power \& Logic Decks):} The lower foundation, designated as the Power Deck, houses the energy source consisting of a 3S Li-Ion battery pack. This deck features a dovetail rail mechanism to facilitate easy removal and maintenance of the batteries. Power is routed through a Battery Management System (BMS) to a female XT60 connector. Above this sits the Logic Deck, which encloses the primary azimuth actuator, a NEMA 17 bipolar stepper motor. This motor bears the structural load of the rotating upper assembly.

\begin{figure}[H]
    \vspace{1em} 
    \centering
    \includegraphics[width=0.6\textwidth]{images/case_design.png}
    \caption{3D design of the hardware enclosure}
    \label{fig:case_design}
\end{figure}

\textbf{Rotating Turret:} The upper rotating platform acts as the mounting chassis for the main controller (ESP32-S3), the elevation actuator (MG90S Servo), and the antenna system. To address mechanical instability, a coupling was employed to interface the motor shaft with the PCB, providing a stable, level plane for the rotor electronics.

\subsubsection{Electromechanical Interface}
To enable continuous 360-degree rotation without cable entanglement, a custom PCB-based slip ring was fabricated. This interface manages the transmission of power and logic signals across the rotating boundary.

A PCB-based slip ring was selected over commercial capsule slip rings to reduce
cost, allow custom signal routing, and enable tight integration with the system
mechanical envelope. The design also permits easy modification of ring width and
spacing to improve contact reliability during iterative prototyping.


\textbf{Stator (Base Side):} The stator consists of a copper-clad board etched with four concentric conductive rings. These rings correspond to the required transmission channels: 5V power, ground,
and digital control signals (STEP and DIR) for the azimuth motor driver.

\begin{figure}[H]
    \vspace{1.5em} 
    \centering
    \includegraphics[width=0.6\textwidth]{images/copper_plate.png}
    \caption{Design of Concentric Copper Plate}
    \label{fig:elect_inteface}
\end{figure}

\textbf{Rotor (Top Side):} The rotating contact mechanism utilizes spring-loaded pogo pins. These are arranged in redundant pairs for each ring to ensure continuous electrical continuity even if one pin momentarily disconnects due to mechanical vibration.

\subsubsection{Electronic Control and Circuit Design}

The electrical system, detailed in the schematic diagram Figure~\ref{fig:circuit_diagram}, is powered by a 12V DC source (via the 3S battery). The voltage regulation and signal flow are designed as follows:

\textbf{Power Distribution:} The regulated 5V rail is transmitted through the slip ring to power the ESP32-S3
and the elevation servo on the rotating platform, while the unregulated 12V supply
is retained in the base to drive the stepper motor coils.

\textbf{Motor Drivers:} High-current motor supply lines were intentionally excluded from the slip ring
interface to minimize electrical noise, contact wear, and voltage drop across
the rotating interface.

\textbf{Azimuth Control:} The NEMA 17 stepper motor is driven by a Pololu A4988 stepper driver (A1). The schematic indicates that the logic control lines (STEP, DIR, ENABLE) and microstepping pins (MS1, MS2, MS3) are connected to the ESP32. Since the driver is located in the base (near the motor) and the ESP32 is on the rotor, the STEP and DIR control signals are generated by the ESP32 and transmitted down through the slip ring to the driver.

\textbf{Elevation Control:} The TowerPro MG90S servo is driven directly by a PWM signal generated by the ESP32-S3's dedicated MCPWM hardware timers. A 100nF decoupling capacitor is placed at the servo power inputs to filter inductive noise and prevent voltage spikes from affecting the microcontroller logic.

\textbf{Microcontroller:} High-current motor supply lines were intentionally excluded from the slip ring
interface to minimize electrical noise, contact wear, and voltage drop across
the rotating interface.


\begin{figure}[H]
    \centering
    \rotatebox{90}{%
        \includegraphics[width=0.9\textheight]{images/circuit_diagram.png}
    }
    \caption{Schematic Diagram of Circuit Connections}
    \label{fig:circuit_diagram}
\end{figure}
