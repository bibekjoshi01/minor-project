\subsection{Background}

Directional antennas are widely used in wireless communication systems to achieve
extended range, improved SNR, and reduced interference by concentrating radiated
energy in a specific direction. Such antennas are commonly deployed in fixed
wireless links, point-to-point backhaul, wireless sensor networks, and long-range
IoT systems. However, the performance benefits of directional antennas are highly
dependent on accurate antenna alignment between the transmitter and receiver.

In practical scenarios, optimal antenna orientation is difficult to maintain due to
environmental changes such as multipath propagation, moving obstacles, structural
vibrations, and gradual displacement of antenna mounts. Even small angular
misalignments can lead to significant degradation in received signal strength and
link reliability. Traditionally, antenna alignment is performed manually during
installation or through periodic maintenance, which is time-consuming, error-prone,
and impractical for systems that require continuous or unattended operation.

Existing automated alignment solutions often rely on expensive hardware, such as
phased antenna arrays, angle-of-arrival estimation, or complex RF front-end
processing. These approaches increase system cost, power consumption, and design
complexity, making them unsuitable for low cost embedded platforms. As a result,
there is growing interest in receiver side autonomous alignment techniques that rely
on readily available signal quality indicators, such as RSSI, throughput, and simple
mechanical steering mechanisms to achieve adaptive directional alignment.